\documentclass[10pt]{article}
%\usepackage[utf8]{inputenc}
\usepackage[font=footnotesize]{caption}
\usepackage{graphicx}
\usepackage{float}
\usepackage{ngerman}
\usepackage{listingsutf8}
\usepackage{booktabs}
\usepackage[version=3]{mhchem}
\usepackage{subfigure} % für Plots nebeneinander
\usepackage{textcomp} % f�r \textmu !!!
\usepackage{dcolumn} % f�r Ausrichtung am Komma


\begin{document}

%\sffamily % TEST Schriftart

\thispagestyle{empty}
\begin{center}

\begin{center}
\includegraphics[width=.3\textwidth]{husiegel_sw_op}\\[0.6cm]
%\textsc{\LARGE HU BERLIN}\\[1.3cm]
\end{center}

{\huge \textbf{E-Praktikum}}\\[6mm]
{\huge \textbf{Operationsverst"arker}}\\[13mm]

\begin{tabbing}
absatzabsatzabsatz\= mehr \kill
\> \textit{Betreuer:}\\
\> Martin Handwerg\\
\> Stefan Weidemann\\[2mm]
\> \textit{Datum:}\\\> 30.06.2016\\[2mm]
%\> \textit{Raum:}\\\> 0'107, Newtonstra"se 15, HU Berlin \\[2mm]
\> \textit{Verfasser:} \\
\> Ian Clotworthy (519006)\\
\> Stefan M"oller (164903)\\[2mm]
\end{tabbing}
\end{center}
\vskip 5ex
\renewcommand
\abstractname{Abstract}
\begin{abstract}

In diesem Versuch wurden Operationsverst"arker aufgebaut und ihre Funktionen wurden 
%Gegenstand dieses Versuchs/von diesem Versuch
%Im Rahmen dieses Versuchs
%(Die) Laserspektroskopie
%Im folgenden (Versuch)
%Mithilfe/Mittels


\end{abstract}

\newpage 

%\rmfamily  %%% TEST Schriftart

\tableofcontents
\newpage

%%%
%%%
%%%

\newpage
\section{Einf"uhrung}
\subsection{Interner Bestandteile}
Intern besteht ein Operationsverst"arker (OPV) aus drei Verst"arkerstufen. Die Eingangsstufe besteht unter anderem aus zwei Transistoren T1 und T2, zwischen deren Basen die Eingangsspannung $U_1$ angelegt wird. Bei $U_1=0V$ ist eine Ausgangsspannung von $U_Q=0V$ w"unschenswert. Die Verst"arkung der ersten Stufe ist rund um $\beta \approxeq 150$.

%%% AUSWERTUNG UND DISKUSSION %%%
\section{Dioden}


\subsection{Strom-Spannungs-Kennlinien} % !!!






\subsection{Differentieller Widerstand einer Diode}


\newpage

\section{Bipolartransistor}
\subsection{Kennlinien}


\subsection{Emitterschaltung ohne Gegenkopplung}


\subsection{Emitterschaltung mit Gegenkopplung}

%\vskip 1cm
\begin{thebibliography}{4}
\bibitem{wei�}O. Chiatti, \textit{Versuchsanleitung Versuch 2: Bipolartransistoren}, (2016)
\bibitem{rot}E. Hering, K. Bressler, J. Gutekunst, \textit{Elektronik f"ur Ingenieure und Naturwissenschaftler}, Springer-Verlag, Berlin (2014)
\bibitem{weinrot}M. Marinescu, J. Winter, \textit{Basiswissen Gleich- und Wechselstromtechnik}, Vieweg, Wiesbaden (2007)
\bibitem{orange}Go"sner, \textit{Grundlagen der Elektronik (Halbleiter, Bauelemente und Schaltungen)}, Shaker Verlag, Aachen (2007) % !!!
\bibitem{transparent}http://www.elektronik-kompendium.de/sites/bau/0201111.htm, 21.06.2016
%\bibitem {schwarz} Dinkelaker, Mandel: Versuchsanleitung im F-Praktikum Physik \textit{Laserspektroskopie an Rubidiumgas}, 2016
%\bibitem {blau} Dr. Uwe M"uller: Physikalisches Grundpraktikum. Einf"uhrung in die Messung. Auswertung und 
%Darstellung experimenteller Ergebnisse in der Physik, 2007
%\bibitem {hellbunt} Steck, D. A. (2010): Rubidium 87 D Line Data; \textit{http://steck.us/alkalidata/rubidium87numbers.pdf}
% \bibitem {bunt} Demtr�der, W. (2010): Experimentalphysik 3 - Atome, Molek�le und Festk�rper (Abschnitt 10.2.7)
%\bibitem {gelb} Demtr�der, W. (2011): Laserspektroskopie 1 (Abschnitt 3.2, Doppler-Verbreiterung und Abschnitt 5.6.1, Halbleiterlaser)

\end{thebibliography}


%%%%%%%%%%%%%%%%%%%%%%%%%%%%%%%%%%%%%%%%%%%%%%%%%%%%%
\begin{appendix}
\section{Aufgenommene Messwerte} % !!!


\begin{table}[h!]
\begin{center}
\caption{Messwerte f"ur die IU-Kennlinien einer Si-Diode}
\begin{tabular}{rr@{,}lr@{,}l}
\toprule
$U_0$/V & \multicolumn{2}{r}{$U_D$/V} & \multicolumn{2}{r}{$U_R$/V}\\
\midrule
2	&0&605&		1&48\\
4	&0&645&		3&43\\
6	&0&665&		5&36\\
8	&0&679&		7&35\\
10	&0&69&		9&38\\
12	&0&698&		11&33\\
14	&0&705&		13&31\\
16	&0&711&		15&27\\
18	&0&716&		17&3\\
20	&0&719&		19&23\\
22	&0&724&		21&2\\
24	&0&727&		23&2\\
26	&0&73&		25&3\\
28	&0&732&		27&2\\
30	&0&734&		29&3\\
\bottomrule
\end{tabular}
\end{center}
\end{table}

\begin{table}[h!]
\begin{center}
\caption{Messwerte f"ur die IU-Kennlinien einer Z-Diode}
\begin{tabular}{rr@{,}lr@{,}l}
\toprule
$U_0$/V & \multicolumn{2}{r}{$U_D$/V} & \multicolumn{2}{r}{$U_R$/V}\\
\midrule
2	&1&947&		0&088\\
4	&3&62&		0&425\\
6	&4&57&		1&455\\
8	&4&89&		3&13\\
10	&5&01&		5&05\\
12	&5&08&		6&99\\
14	&5&11&		8&91\\
16	&5&14&		10&88\\
18	&5&16&		12&81\\
20	&5&17&		14&8\\
22	&5&19&		16&81\\
24	&5&2&		18&78\\
26	&5&21&		20&8\\
28	&5&22&		22&8\\
30	&5&23&		24&7\\
\bottomrule
\end{tabular}
\end{center}
\end{table}

\begin{table}[h!]
\begin{center}
\caption{Messwerte f"ur die IU-Kennlinien einer LED}
\begin{tabular}{D{,}{,}{1}D{,}{,}{-1}D{,}{,}{1}}
\toprule
\multicolumn{1}{r}{$U_0$/V} & \multicolumn{1}{r}{$U_D$/V} & \multicolumn{1}{r}{$U_R$/mV}\\
\midrule
0,2&	0,202&		0\\
0,5&	0,539&		0\\
0,8&	0,785&		0\\
0,9&	0,928&		0,1\\
1&	1,07&		0,1\\
1,1&	1,165&		0,1\\
1,2&	1,229&		0,2\\
1,3&	1,348&		1,8\\
1,4&	1,436&		12,9\\
1,5&	1,489&		47,1\\
1,6&	1,532&		132\\
1,7&	1,55&		206\\
1,8&	1,557&		244\\
1,9&	1,574&		374\\
2&	1,584&		469\\
2,1&	1,592&		554\\
2,2&	1,597&		631\\
2,4&	1,611&		848\\
2,6&	1,619&		993\\
2,8&	1,629&		1209\\
3&	1,639&		1445\\
3,2&	1,646&		1641\\
3,4&	1,65&		1786\\
3,6&	1,656&		1974\\
3,8&	1,661&		2170\\
4&	1,667&		2370\\
4,2&	1,673&		2610\\
4,4&	1,677&		2810\\
4,6&	1,682&		3000\\
4,8&	1,685&		3150\\
5&	1,69&		3400\\
\bottomrule
\end{tabular}
\end{center}
\end{table}

\begin{table}[h!]
\begin{center}
\caption{Messwerte f"ur die Ermittlung des differentiellen Widerstands einer Si-Diode bei einer angelegten Wechselspannung von $u_{0,ac}$=50\,mV$_\textrm{rms}$}
\begin{tabular}{rD{,}{,}{-1}D{,}{,}{1}D{,}{,}{1}}
\toprule
\multicolumn{1}{r}{$U_{0,dc}$/V} & \multicolumn{1}{r}{$U_R$/V} & \multicolumn{1}{r}{$u_{ac}$/mV$_\textrm{rms}$} &  \multicolumn{1}{r}{$u_D$/mV$_\textrm{rms}$}\\
\midrule
5 & 4,41 & 52,3 & 6,2\\
10 & 9,45 & 54,1 & 3,1\\
15 & 14,42 & 54,7 & 2\\
20 & 19,39 & 55 & 1,8\\
\bottomrule
\end{tabular}
\end{center}
\end{table}

\begin{table}[h!]
\begin{center}
\caption{Messwerte f"ur die Ermittlung des differentiellen Widerstands einer Si-Diode bei einer angelegten Wechselspannung von $u_{0,ac}$=500\,mV$_\textrm{rms}$}
\begin{tabular}{rD{,}{,}{-1}D{,}{,}{1}D{,}{,}{1}}
\toprule
\multicolumn{1}{r}{$U_{0,dc}$/V} & \multicolumn{1}{r}{$U_R$/V} & \multicolumn{1}{r}{$u_{ac}$/mV$_\textrm{rms}$} &  \multicolumn{1}{r}{$u_D$/mV$_\textrm{rms}$}\\
\midrule
5 & 4,58 & 438 & \approx300\\
10 & 9,39 & 554 & \approx40\\
15 & 14,33 & 564 & \approx20\\
20 & 19,38 & 568 & \approx15\\
\bottomrule
\end{tabular}
\end{center}
\end{table}

% 3.3.3 Tabelle
%\begin{table}
%\begin{center}
%\caption{lorem ipsum}
%\begin{tabular}{D{1}{,}{,}D{1}{,}{,}D{1}{,}{,}}
%\toprule
%\multicolumn{1}{r}{$f$/Hz} & \multicolumn{1}{r}{$U_e$/mV$_{\textrm{pp}}$} & \multicolumn{1}{r}{$U_a$/V$_{\textrm{pp}}$}\\
%\multicolumn{1}{r}{$f$/Hz} & \multicolumn{1}{r}{$U_e$} & \multicolumn{1}{r}{$U_a$}\\
%\midrule
%1,0 & 12,0 & 0,24\\
%10 & 12,8 & 0,40\\
%20 & 14,4 & 0,96\\
%100 & 30 & 2,68\\
%1\,k & 28,8 & 2,74\\
%10\,k & 33,6 & 2,74\\
%20\,k & 30 & 2,72\\
%1\,M & 25,6 & 2,08\\
%2\,M & 22,6 & 1,24\\
%2,5\,M & 24,2 & 1,06\\
%3\,M & 22,4 & 0,9\\
%\bottomrule
%\end{tabular}
%\end{center}
%\end{table} % !!!


% !!!
% 3.4 Tabelle stichprobenartige �berpr�fung der Verst�rkung
\begin{table}
\begin{center}
\caption{stichprobenartige "Uberp"ufung der Verst"arkung}
\begin{tabular}{rrr}
\toprule
$f$/Hz & $U_e$/mV$_{\textrm{pp}}$ & $ U_a$/mV$_{\textrm{pp}}$\\
\midrule
10 & 9 & 80\\
30 & 25 & 250\\
50 & 20 & 310\\
100 & 25 & 340\\
100000 & 25 & 320\\
\bottomrule
\end{tabular}
\end{center}
\end{table} % !!!

\end{appendix}

\end{document}


%%%   %      %   %%%    %%% 
%         %%   %   %    %   %
%%%   % %  %   %     %  %%%
%         %   %%   %     %  %
%%%   %      %   %%%    %%%


%%% Grafik %%%
\begin{figure}[h!]
  \begin{center}
    \includegraphics[width=100mm]{}
    \caption{}
   \end{center}
\end{figure}


%%% Tabelle %%% 
\begin{table}[h!]
\begin{center}
\begin{tabular}{rrrrrr}
\toprule
$\theta [^\circ] $ & $\psi [^\circ]$ & $\Delta [^\circ]$ & $n_W$[\textendash] & n [\textendash] & $\kappa$ [\textendash] \\
\midrule
46 & 11,47 & 180,33 & 1,335 & 1,335 & 0,002 \\
50 & 5,18 & 179,53 & 1,337 & 1,336 & 0,001 \\
56 & 4,62 & 0,34 & 1,335 & 1,334 & 0,001 \\
60 & 10,53 & 6,04 & 1,346 & 1,346 & 0,032 \\
\bottomrule
\end{tabular}
\caption{Brechungsindex und Extinktionskoeffizient von Wasser, ermittelt "uber die Messung von Ellipsometerwinkel $\psi$ und Phasenverschiebung $\Delta$ eines an der Grenzfl"ache Luft-Wasser reflektierten Laserstrahls f"ur verschiedene Einfallswinkel $\theta$. Berechnung der Werte in Spalte vier nach Formel (3), in den letzten beiden Spalten durch die auf dem am Arbeitsplatz befindlichen PC installierte Analyse-Software.}
\end{center}
\end{table}


\\\\
